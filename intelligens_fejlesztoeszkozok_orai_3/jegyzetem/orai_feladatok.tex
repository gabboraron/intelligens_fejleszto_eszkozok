\documentclass{article}
\usepackage[utf8]{inputenc}
\usepackage{graphicx}
\usepackage{amsmath}

\title{Intelligens Fejlesztőeszkozok - 2. órai jegyzet}
\author{Burian Sándor}
\date{Szeptember 2022}

\begin{document}

\maketitle

\section{feladat}
\begin{equation}
   \begin{cases}
      y'-\frac{2xy}{x^{2}+1}=x^{3}+x\\
	  y(0)=1
    \end{cases}       
\end{equation}

I.

\begin{multline}
\\
y'-\frac{2xy}{x^{2}+1} = 0 \\
\frac{dy}{dx} = \frac{2xy}{x^{2}+1} \\
\int \frac{1}{4}dy = \int \frac{2x}{x^{2}+1}\\
ln|y| = ln|x^{2}+1|+ln|c| \\
yn = c|x^{2}+1 \\
\end{multline}

II.

\begin{multline}
\\
yp=k(x)(x^{2}+1) \\
yp=k'(x)(x^{2}+1)+k(x)2x \\
k'(x)(x^{2}+1)-k(x)2x-\frac{2xkx(x^{2}+1}{x^{2}+1} = x(x^{2}+1) \\
\end{multline}

\begin{multline}
\\
k(x)=x \\
\int k'(x)dx = \int x dx \\
k(x) = \frac{x^{2}}{2} \\
yp = \frac{x^{2}}{2}(x^{2}+1)=\frac{x^{4}+x^{2}}{2} \\
\end{multline}

III.

\begin{multline}
\\
y = Yn + yp = (x^{2}+1) + \frac{x^{4}+x^{2}}{2} \\
1= c \\
y_{imp} =  (x^{2}+1) + \frac{x^{4}+x^{2}}{2} \\
\end{multline}


\section{feladat}

\begin{equation}
y"-4y'+4y=12x-4      
\end{equation}

I.
\begin{multline}
\\
y"-4y'+4y=0 \\
\lambda^{2}-4\lambda+4 = 0 \\
\lambda_{1,2} = \frac{4\pm \sqrt{16-16}}{2}=2 \\
yh= c_{1} e^{2x}+c_{2} e^{2x} \\
\end{multline}

II.
\begin{multline}
\\
yp = Ax+B \\
y'p = Ax \\
y"p = 0 \\
\\
-4Ax+4B = 12x-4 \\
A = 3 \\
b = 2 \\
\Rightarrow yp = 3x+2
\end{multline}

III.

\begin{multline}
\\
y = c_{1}e^{2x}+c_{2}xe^{2x}+3x+2 \\
y' = c_{2}e^{2x}+c_{2}e^{2x}+2c_{2}xe^{2x}+3 \\
3 = c_{1}+2 \\
8 = 2+c_{2}+3 \Rightarrow c_{2}=3 \\
\end{multline}

\begin{equation}
y_{imp} = e^{2x}+3xe^{2x}+3x+2
\end{equation}

\section{feladat}

\begin{equation}
   \begin{cases}
    y' = f(x,y) \\
    y(x_{0}=y_{0}
    \end{cases}
\end{equation}

\subsection{Euleres megoldása}

\begin{equation}
   \begin{cases}
	x_{n+1}=x_{n}+h \\
	y_{n+1}-y_{n} = hAn \\
	y_{n+1} = y_{n} +hAn \\
	An = f(x_{n},y_{n})\\
    \end{cases}
\end{equation}

\begin{equation}
   \begin{cases}
    y' = x^2-y^2 \\
    y_{0} = 1 \\
    h = 0,1 \\
    \end{cases}
\end{equation}
2 lépés:
\begin{center}
\begin{tabular}{ c c c c c }
 n & $x_{n}$ & $y_{n}$ & An & hAn \\ 
 0 & 0 & 1 & -1 & -0,1  \\  
 1 & 0,1 & 0,9 & -0,8 & -0,08 \\
 2 & 0,2 & 0,82 & & \\    
\end{tabular}
\end{center}

\subsection{Runge-Kutta}

\begin{equation}
   \begin{cases}
    x_{n+1} = x_{n}+h \\
    \rightarrow An - f(An, yn) \\
    \rightarrow \hat{y_{n+1}} = y_{n}+hAn \\
    \rightarrow B_{n} = f(y_{n+1}, \hat{Y}_{n+1}) \\
    \rightarrow y_{n+1} = y_{n}+h(\frac{A_{n}+B_{n}}{2})
    \end{cases}
\end{equation}

\subsection{RKM}

\begin{equation}
   \frac{A_{n}+2B_{n}+2C_{n}+B_{n}}{6}
\end{equation}

\end{document}
