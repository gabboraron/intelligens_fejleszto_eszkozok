\documentclass{article}
\usepackage[utf8]{inputenc}
\usepackage{graphicx}
\usepackage{amsmath}

\title{Intelligens Fejlesztőeszkozok - 4. beadandó}
\author{Sándor Burian}
\date{Szeptember 2022}

\begin{document}

\maketitle

\section{feladat}

Intervallum felező módszerrel \footnote{https://www.uni-miskolc.hu/~matgt/pdf/nummod/NumMod.pdf (elérés 2022-10-01) alapján}
\begin{equation}
    5x-4 = sin(tanh(-3x+2)); [-10,10] intervallumon 
\end{equation}

\begin{multline}
\\
f(x) = 5x-4-sin(tanh(-3x+2)) \\
f(-10)=-50-4-sin(tanh(32)) = -54 - 0,017452406 = -54,017452406 \\
f(10) = 50-4 - sin(tanh(-28)) = 46 - (-0,017452406) = 46,017452406 \\
\\
\Rightarrow f(a)f(b)<0 \Rightarrow intervallumfelezes: \dfrac{a+b}{2} \Rightarrow \frac{10+(-10)}{2} = 0 \\
\end{multline}

\begin{multline}
\\
f(0) = -4-sin(tanh(2)) = -4-0,016824661 = -4,016824661\\
\Rightarrow f(-10)f(0) > 0 \Rightarrow [0,10] intervallumot felezzuk \Rightarrow [0,+5] \\
\end{multline}

\begin{multline}
\\
f(5) = 25-4-sin(tanh(-13)) = 21,017452406 \\
f(0)f(5) < 0 \Rightarrow uj intervallum [0,2.5] \\
f(2.5) = 12.5-4-sin=tanh(-5.5)) = 8,517451824 \\
\Rightarrow f(0)f(2.5) < 0 uj intervallum: [0, 1.25] \\
f(1.25) = 6.25-4-sin(tanh(-1.75)) = 2,266429363 \\
\Rightarrow f(0)f(1.25) < 0 \Rightarrow uj intervallum [0, 0.625] \\
f(0.625) = 3.125-4-sin=tanh(0.125)) = -0,877170368 \\
\Rightarrow f(0.625)f(10) < 0 \Rightarrow ujintervallum [0.625,10] felezese [0.625,5.3125]\\
f(5.3125) = 26.5625-4-sin(tanh(-13.9375)) = 22,017452406 \\
\Rightarrow \dfrac{0.625+5.3125}{2} = 2.96875 \Rightarrow [0.625, 2.96875] \\
f(2.96875) = 14.84375 - 4 - sin(tanh(-6.90625)) = 10.861202371 \\
\Rightarrow uj intervallum [0.625, 1.796875] \\
f(1.796875) = 8.984375-4-sin(tanh(-3.3903625)) = 5.001787843 \\
\Rightarrow uj intervallum [0.625, 0.91796875] \\
f(0.91796875) = = 4.58984375 - 4 -sin(tanh(-0.314453125)) = -0.137263922\\
\Rightarrow uj intervallum [0.625, 0.771484375]\\
f(0.771484375) = 3.857421875 - 4 -sin(tanh(-0.314453125)) = -0.137263922\\
\\
\Rightarrow f(0.625)f(771484375)>0
\end{multline}

Mivel a két függvényérték szorzata nem negatív, ezért leállunk. A függvény zéruspontja a [0.625, 771484375]. f(0.625) = -0,877170368 és f(0.771484375) = -0.137263922 küzül az utóbbi van közelebb a 0-hoz, így látható, hogy 0.771484375 pontról van szó.

\section{feladat}

\begin{equation}
x+3= e^{sin(x+3)}
\end{equation}

Intervallum felező módszerrel [-10,10] között.

\begin{multline}
\\
f(x) = x+3-e^{sin(x+3)} \\
f(-10) = -7 -e^{0.121869343} = -7.885264027\\
f(10) = 13-e^{0.224951054} = 11.747738578\\
\Rightarrow f(-10)f(10) < 0 \Rightarrow \dfrac{10+10}{2} = 0 \\
\\
f(0) = 3-e^{0.052335956} = 1.94627031\\
\Rightarrow [-10,0] \rightarrow [-10,-5]\\
f(-5) = -2-e^{0.034899497} = 2.965702467\\
\Rightarrow uj intervalum [-10,-7.5] \\
f(-7.5) = -4.5 - e^{0.078459096} = -5.424539877 \\
\Rightarrow f(-10)f(-7.5)>0 \Rightarrow intervallumvaltas [-7.5,-5] \rightarrow -6.25\\
f(-6.25)=-3.25-e^{0.108866875}=-4.146849803\\
\Rightarrow [-6.25,-5] intervallum felezese -5.625\\
f(-5.625)=-2.625-e^{0.09801714}=-2.906633364\\
\Rightarrow [-5.625,-5] fele -5.3125\\
f(-5.3125) = -2.3125-e^{0.040349782} = -3.272953431\\
\Rightarrow [-5.3125, -5] fele -5.15625\\
f(-5.15625) = -2.15625-e^{-0.037624779}=-3.119324239\\
\Rightarrow [-5.15625,-5] fele -5.078125\\ 
f(-5.078125) = -2.078125-e^{-0.036262172}=-3.042512425\\
\Rightarrow [-5.078125,-5] fele -5.0390625\\
\end{multline}
Tehát egyértelműen tart \textit{-5}-be a végtelenben.

\section{feladat}

\begin{equation}
6x+3 = tanh(tan(cos(-4x^{2}-3)))
\end{equation}

[-10,10] tartományon, Húr módszerrel

Húr engyelete:
\begin{multline}
y-f(a)=\dfrac{f(b)-f(a)}{b-a}(x-a)\\
x=\dfrac{af(b)-bf(a)}{f(b)-f(a)}\\
\end{multline}
Legyen c (c, 0) pont az OX tengely metszéspontja a húron.

Ekkor:
\begin{multline}
f(-10)=-60+3-tanh(tan(cos(-403))) = -57.01276453\\
f(10) = 60+3-tanh(tan(cos(-403))) = 62.98723547\\
\end{multline}

A(-10, -57.01276453)) és B(10, 62.98723547) húr és OX tengely metszéspontját keressük.

\begin{multline}
c = \dfrac{-10*62.98723547-10*(-57.01276453)}{62.98723547+57.01276453} =0.497872578\\
f(-0.497872578) = -6 \dot{•} 0.497872578 + 3 -0.017410957 = -0.004646425\\
\end{multline}

Tehát A'(-0.497872578, -0.004646425) és B pontokat összekötő húr OX metszéspontját:

\begin{multline}
c' = \dfrac{0.497872578*62.987235457-10(-0.004646425)}{62.987235457 + 0.004646425} = -0.497098231\\
f(-0.497098231) = -6*0.497098231+3-0,017411023 = -0.000000409\\
\end{multline}


\section{feladat}
\begin{equation}
x+2 = x^{3}
\end{equation}

[-10,10] tartományon, Newton-Raphson módszerrel

\begin{multline}
f(x) = x+2 - x^{3}\\
f'(x) = 1-3x^{2}\\
f''(x)=-6x\\
\end{multline}

\begin{multline}
f(-10) = -8+1000= 992\\
f'(-10) = 1 - 300 = -299\\
f"(-10) = 60\\
f(10) = 12-1000 = -988\\
x1 = \dfrac{-10 - 992 }{-299} = -10+3.317725753 = -6.682274247\\
f( -6.682274247)= 293,699908494\\
x2 = -6.682274247 - \dfrac{f(-6.682274247)}{ f'(-6.682274247)} = -4,473312793\\
f(-4,473312793)=87,04003517\\
x3 = - 4,473312793 - \dfrac{f(-4,473312793)}{f'(-4,473312793)}= -2,998847224\\
f(-2,998847224)=25,970039783\\
x4 = -2,998847224 - \dfrac{f(-2,998847224) }{f'(-2,998847224) }= -1,999201901\\
f(-1,999201901)=7,991224732\\
x5 = -1,999201901 - \dfrac{f(-1,999201901)}{f'(-1,999201901)}= -1,272093993\\
f(-1,272093993)=2,786437926\\
x6 = -1,272093993 - \dfrac{f(-1,272093993)}{f'(-1,272093993)} = -0,549220602\\
f(-0,549220602)=1,616448096\\
x7 = -0,549220602 - \dfrac{f(-0,549220602)}{f'(-0,549220602)} =-1,397781052\\
f(-1,397781052) = 3,333192202\\
\end{multline}
Az előző pontosabb, így leállási feltétel teljesül f(-0,549220602)=1,616448096

\section{feladat}
\begin{equation}
f(x)= sin(x-5)
\end{equation}

[-10,10] tartományon, Fixpont iterációval 

\begin{multline}
f(x)=0\\
sin(x-5)=0\\
sin(x)cos(5)-cos(x)sin(5)=0\\
\end{multline}

jelöljük:
\begin{multline}
y=sin(x)\\
sin^{2}(x)+cos^{2}(x)=1 \Rightarrow cos^{2}(x) = 1-y^{2}\\
\end{multline}

\begin{multline}
sin(x)cos(5)=cos(x)sin(5) \\
y^{2}cos^{2}(5) = (1-y^{2})sin^{2}(5)\\
y^{2}cos^{2}(5)+y^{2}sin^{2}(5) = sin^{2}(5) \leftarrow sin^{a}+cos^{2}(a) = 1 \\
y^{2}(cos^{2}(5)+sin^{2}(5)) = sin^{2}(5)\\
y^{2} = sin^{2}(5) \leftarrow y=sin(x) \\
sin(x)=sin(5) \Rightarrow x=5
\end{multline}

a függvény x+g(x) alakban:

\begin{multline}
sin(x-5)=x-5-\frac{(x-5)^{3}}{3!}+\frac{(x-5)^{5}}{5!} - \frac{(x-5)^{7}}{7!} + ....
x_{0} = 5.5\\
x_{1} = 0.5-\dfrac{0.125}{6}+\dfrac{0.03125}{120}-\dfrac{0.0078125}{5040} = 0.479425533\\
x_2 = 0.479425533-5 - \dfrac{(0.479425533-5)^3}{6} +\dfrac{(0.479425533-5)^5}{120} - \dfrac{0.479425533-5)^7}{5040} = 2.798730405\\
\Rightarrow x_2 - 5 = -2.201269595\\
x_3 = 2,798730405 - 5 - \dfrac{-2,201269595^3}{6}+\dfrac{-2,201269595^5}{120} - \dfrac{-2,201269595^7}{5040} =-0,804547209\\
\Rightarrow x_3 - 5 = -5,804547209\\
\end{multline}

\section{feladat}
\begin{equation}
f(x)= cos(x-6) =0
\end{equation}

\begin{multline}
cos(x)cos(6)+sin(x)sin(6) = 0\\
cos(x)cos(6)=-sin(x)sin(6) = 0\\
cos^{2}(x)cos^{2}(6)=sin^{2}(x)sin^{2}(6)\\
(1-sin^{2}(x))cos^{2}(6) = sin^{2}(x)sin^{2}(6)\\
sin^{2}(x)(sin^{2}(x)cos^{2}(6) = 1) \\
sin^{2}(x)= 1 \\
x = \pm 90\\
x_0 = 2 \\
x_1 = 1-(-4)^{2}/2 +(-4)^{4}/4! - (-4)^{6}/6! + (-4)^{8}/8! = -0,396825397\\
x_2 = f(-0,396825397-6)= 24,680488403\\
x_3 = f(24,680488403-6) = 313658,348483175\\
\end{multline}




\begin{equation}
   \begin{cases}
      5x-4 = sin(tanh(-3x+2))
    \end{cases}       
\end{equation}
 
\end{document}
